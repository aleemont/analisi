\documentclass{article}
\usepackage{amsmath}
\usepackage{ragged2e}

\title{Analisi Matematica}
\author{Alessandro Monticelli}
\date{A.A. 2021/2022}
  
\begin{document}
  
\maketitle
\newpage
\tableofcontents
\newpage
\section*{Introduzione}
    Appunti di Analisi matematica - corso di Ingegneria e Scienze Informatiche.
\addcontentsline{toc}{section}{Introduzione}
\newpage
\section{Insiemi}
\subsection{Definizione}
    Un insieme è una collezione di elementi. Per ogni elemento si può dire se esso appartiene all'insieme, o no.
\paragraph{Notazioni:}Un \textbf{insieme} si esprime con una \textbf{lettera maiuscola} \{A,B,C,...\}, un \textbf{elemento} si esprime con una \textbf{lettera minuscola}\{a,b,c,...\}.

\subsection{Concetti di base e operatori}
\subsection{Inclusione}
    \begin{LARGE}
        \begin{equation*}
            A \subseteq B
        \end{equation*}
    \end{LARGE}
    Tutti gli elementi di A appartengono a B\newline
    
    \subsubsection*{Esempio:}
    
    \begin{LARGE}
        \begin{equation*}
            A = \{2,5,6,7\}
        \end{equation*}
        \begin{equation*}
            B = \{1,2,3,4,5,6,7,8,9\}
        \end{equation*}
        \begin{equation*}
            A \subseteq B
        \end{equation*}
    \end{LARGE}
    Il sottoinsieme si dice \textit{improprio} se A coincide con B, altrimenti si dice \textit{proprio}.

\subsection{Unione}
    \begin{LARGE}
        \begin{equation*}
            A \cup B
        \end{equation*}
    \end{LARGE}
    Tutti gli elementi del primo insieme e tutti gli elementi del secondo\newline
    
    \subsubsection*{Definizione:}
    
    \begin{LARGE}
        \begin{equation*}
            A \cup B = \{x | x \in A \vee x \in B\}
        \end{equation*}
    \end{LARGE}
    \addcontentsline{toc}{section}{Insiemi}

\end{document}