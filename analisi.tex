\documentclass{article}
\usepackage{amssymb}
\usepackage{amsmath}

\title{Analisi Matematica}
\author{Alessandro Monticelli}
\date{A.A. 2021/2022}
  
\begin{document}
  
\maketitle
\newpage
\tableofcontents
\newpage
\section*{Introduzione}
    Appunti di Analisi matematica - corso di Ingegneria e Scienze Informatiche.
\addcontentsline{toc}{section}{Introduzione}
\newpage
\section{Insiemi}
    \subsection{Definizione}
        Un insieme è una collezione di elementi. Per ogni elemento si può dire se esso appartiene all'insieme, o no.
    \paragraph{Notazioni:}Un \textbf{insieme} si esprime con una \textbf{lettera maiuscola} \{A,B,C,...\}, 
        un \textbf{elemento} si esprime con una \textbf{lettera minuscola}\{a,b,c,...\}.

    \subsection{Concetti di base e operatori}
    \subsubsection{Inclusione}
        \begin{LARGE}
            \begin{equation*}
                A\ \subseteq\ B
            \end{equation*}
        \end{LARGE}
        Tutti gli elementi di A appartengono a B\newline
        
        \subsubsection*{Esempio:}
        
        \begin{LARGE}
            \begin{equation*}
                A\ =\ \{2,5,6,7\}
            \end{equation*}
            \begin{equation*}
                B\ =\ \{1,2,3,4,5,6,7,8,9\}
            \end{equation*}
            \begin{equation*}
                A\ \subseteq\ B
            \end{equation*}
        \end{LARGE}
        Il sottoinsieme si dice \textit{improprio} se A coincide con B, altrimenti si dice \textit{proprio}.

    \subsubsection{Unione}
        \begin{LARGE}
            \begin{equation*}
                A\ \cup\ B
            \end{equation*}
        \end{LARGE}
        Tutti gli elementi del primo insieme e tutti gli elementi del secondo\newline
        
        \subsubsection*{Definizione:}
        
        \begin{LARGE}
            \begin{equation*}
                A\ \cup\ B\ =\ \{x\ |\ x\ \in\ A\ \vee\ x\ \in\ B\}
            \end{equation*}
        \end{LARGE}

        \subsubsection{Intersezione}
        \begin{LARGE}
            \begin{equation*}
                A\ \cap\ B
            \end{equation*}
        \end{LARGE}
        Tutti gli elementi comuni al primo e al secondo insieme\newline
        
        \subsubsection*{Definizione:}
        
        \begin{LARGE}
            \begin{equation*}
                A\ \cap\ B\ =\ \{x\ |\ x\ \in\ A\ \wedge\ x\ \in\ B\}
            \end{equation*}
        \end{LARGE}

        \subsubsection{Differenza}
        \begin{LARGE}
            \begin{equation*}
                A \backslash B
            \end{equation*}
        \end{LARGE}
        Elementi appartenenti \textbf{solo} ad A e non a B\newline
        
        \subsubsection*{Definizione:}
        
        \begin{LARGE}
            \begin{equation*}
                A\ \backslash\ B\ =\ \{x\ |\ x\ \in\ A\ \wedge\ x\ \notin\ B\}
            \end{equation*}
        \end{LARGE}

        \subsubsection*{Osservazione:}

        \begin{Large}
            \begin{equation*}
                A\ \backslash\ B\ \ne\ B\ \backslash\ A
            \end{equation*}
        \end{Large}

        \subsubsection{Differenza Simmetrica}
        \begin{LARGE}
            \begin{equation*}
                A\ \bigtriangleup\ B
            \end{equation*}
        \end{LARGE}
        
        \subsubsection*{Definizione:}
        
        \begin{LARGE}
            \begin{equation*}
                A\ \bigtriangleup\ B\ =\ (A\ \backslash\ B)\ \cup\ (B\ \backslash\ A)
            \end{equation*}
        \end{LARGE}

        \subsubsection*{Osservazione:}

        \begin{Large}
            \begin{equation*}
                A\ \bigtriangleup\ B\ =\ B\ \bigtriangleup\ A
            \end{equation*}
        \end{Large}

        \subsubsection{Prodotto Cartesiano}
        \begin{LARGE}
            \begin{equation*}
                A \times B
            \end{equation*}
        \end{LARGE}
        
        \subsubsection*{Definizione:}
        
        \begin{LARGE}
            \begin{equation*}
                A\ \times\ B = \{(a,b)\ |\ a\ \in\ A\ \wedge\ b\ \in\ B\}
            \end{equation*}
        \end{LARGE}

        \subsubsection*{Osservazione:}

        \begin{Large}
            \begin{equation*}
                (a,b)\ \ne\ (b,a)\ \Rightarrow\ A\ \times\ B\ \ne\ B\ \times\ A
            \end{equation*}
        \end{Large}

        \subsubsection{Insieme Vuoto}
            \textbf{Notazione:}\newline
            \begin{LARGE}
                \begin{equation*}
                    A = \oslash
                \end{equation*}
            \end{LARGE}
        
\section{Proposizioni}
    \subsection{Definizione}
        Una proposizione è un'affermazione che è falsa o vera e che può implicare altre affermazioni.
        \newpage
        Con $p, q$ proposizioni:

        \begin{LARGE}
            \begin{equation*}
                p \Rightarrow q
            \end{equation*}
        \begin{equation*}
              \Downarrow
        \end{equation*}
        \begin{equation*}
              p\ implica\ q
        \end{equation*}
        \end{LARGE}
        Se $p$ \textit{implica} $q$ e $q$ \textit{implica} $p$ si dicono \textit{equivalenti}\newline
        \begin{LARGE}
            \begin{equation*}
                p\ \iff\ q
            \end{equation*}
        \end{LARGE}
        \subsection{Quantificatori}
            \begin{itemize}
                \item $\forall$ - per ogni
                \item $\exists$ - esiste
                \item $\exists!$ - esiste ed è unico
                \item $\nexists$ - non esiste
            \end{itemize}
\end{document}