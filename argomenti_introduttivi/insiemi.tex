\section{Insiemi}
    \subsection{Definizione}
        Un insieme è una collezione di elementi. Per ogni elemento si può dire se esso appartiene all'insieme, o no.
    \paragraph{Notazioni:}Un \textbf{insieme} si esprime con una \textbf{lettera maiuscola} \{A,B,C,...\}, 
        un \textbf{elemento} si esprime con una \textbf{lettera minuscola}\{a,b,c,...\}.

    \subsection{Concetti di base e operatori}
    \subsubsection{Inclusione}
        \begin{LARGE}
            \begin{equation*}
                A\ \subseteq\ B
            \end{equation*}
        \end{LARGE}
        Tutti gli elementi di A appartengono a B\newline
        
        \subsubsection*{Esempio:}
        
        \begin{LARGE}
            \begin{equation*}
                A\ =\ \{2,5,6,7\}
            \end{equation*}
            \begin{equation*}
                B\ =\ \{1,2,3,4,5,6,7,8,9\}
            \end{equation*}
            \begin{equation*}
                A\ \subseteq\ B
            \end{equation*}
        \end{LARGE}
        Il sottoinsieme si dice \textit{improprio} se A coincide con B, altrimenti si dice \textit{proprio}.

    \subsubsection{Unione}
        \begin{LARGE}
            \begin{equation*}
                A\ \cup\ B
            \end{equation*}
        \end{LARGE}
        Tutti gli elementi del primo insieme e tutti gli elementi del secondo\newline
        
        \subsubsection*{Definizione:}
        
        \begin{LARGE}
            \begin{equation*}
                A\ \cup\ B\ =\ \{x\ |\ x\ \in\ A\ \vee\ x\ \in\ B\}
            \end{equation*}
        \end{LARGE}

        \subsubsection{Intersezione}
        \begin{LARGE}
            \begin{equation*}
                A\ \cap\ B
            \end{equation*}
        \end{LARGE}
        Tutti gli elementi comuni al primo e al secondo insieme\newline
        
        \subsubsection*{Definizione:}
        
        \begin{LARGE}
            \begin{equation*}
                A\ \cap\ B\ =\ \{x\ |\ x\ \in\ A\ \land\ x\ \in\ B\}
            \end{equation*}
        \end{LARGE}

        \subsubsection{Differenza}
        \begin{LARGE}
            \begin{equation*}
                A \setminus B
            \end{equation*}
        \end{LARGE}
        Elementi appartenenti \textbf{solo} ad A e non a B\newline
        
        \subsubsection*{Definizione:}
        
        \begin{LARGE}
            \begin{equation*}
                A\ \setminus\ B\ =\ \{x\ |\ x\ \in\ A\ \land\ x\ \notin\ B\}
            \end{equation*}
        \end{LARGE}

        \subsubsection*{Osservazione:}

        \begin{Large}
            \begin{equation*}
                A\ \setminus\ B\ \ne\ B\ \setminus\ A
            \end{equation*}
        \end{Large}

        \subsubsection{Differenza Simmetrica}
        \begin{LARGE}
            \begin{equation*}
                A\ \bigtriangleup\ B
            \end{equation*}
        \end{LARGE}
        
        \subsubsection*{Definizione:}
        
        \begin{LARGE}
            \begin{equation*}
                A\ \bigtriangleup\ B\ =\ (A\ \setminus\ B)\ \cup\ (B\ \setminus\ A)
            \end{equation*}
        \end{LARGE}

        \subsubsection*{Osservazione:}

        \begin{Large}
            \begin{equation*}
                A\ \bigtriangleup\ B\ =\ B\ \bigtriangleup\ A
            \end{equation*}
        \end{Large}

        \subsubsection{Prodotto Cartesiano}
        \begin{LARGE}
            \begin{equation*}
                A \times B
            \end{equation*}
        \end{LARGE}
        
        \subsubsection*{Definizione:}
        
        \begin{LARGE}
            \begin{equation*}
                A\ \times\ B = \{(a,b)\ |\ a\ \in\ A\ \land\ b\ \in\ B\}
            \end{equation*}
        \end{LARGE}

        \subsubsection*{Osservazione:}

        \begin{Large}
            \begin{equation*}
                (a,b)\ \ne\ (b,a)\ \Rightarrow\ A\ \times\ B\ \ne\ B\ \times\ A
            \end{equation*}
        \end{Large}

        \subsubsection{Insieme Vuoto}
            \textbf{Notazione:}\newline
            \begin{LARGE}
                \begin{equation*}
                    A = \oslash
                \end{equation*}
            \end{LARGE}
 