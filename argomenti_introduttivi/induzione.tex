\subsection{Principio di induzione}
        \subsubsection*{Teorema}
            Sia $p(n)$ un insieme di proposizioni al variare di $n\ \in\ \mathbb{N}$.\\
            Supponiamo che:\\
            \begin{Large}
                \begin{itemize}
                    \item $p(0)$ sia vera
                    \item $\forall\ n\ \in\ \mathbb{N}, p(n)$ vera $\Rightarrow\ p(n\ +\ 1)$ vera. 
                \end{itemize}
            \end{Large}
        \subsubsection*{Esempio}
        Dimostrare:
        \begin{Large}
            \[1+2+3+\cdots+n \Rightarrow \sum_{k=1}^{n}k = \frac{n(n+1)}{2}\] 
        \end{Large}
        \begin{Large}
            \paragraph{Dimostrazione per induzione: \\}
            \begin{equation}
                p(1) \Rightarrow \frac{1(2)}{2} = 1 \Rightarrow vera
            \end{equation}
            \begin{equation}
                    \sum_{k=1}^{n+1}k = \frac{(n+1)(n+2)}{2}\ ?
            \end{equation}
            \begin{equation*}
                \sum_{k=1}^{n+1}k = 1+2+3+\cdots+n+(n+1) = \frac{n(n+1)}{2}+(n+1)=
            \end{equation*}
            \begin{equation*}
                =n+1(\frac{n}{2}+1)=n+1(\frac{n+2}{2})=\frac{(n+1)(n+2)}{2}\ \Rightarrow\ p(n+1)\ vera
            \end{equation*}
            \hfill \break
        \end{Large}
            La proposizione è verificata.
\