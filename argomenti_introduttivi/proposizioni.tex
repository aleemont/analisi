\section{Proposizioni}
    \subsection{Definizione}
        Una proposizione è un'affermazione che è falsa o vera e che può implicare altre affermazioni.
        Con $p, q$ proposizioni:

        \begin{LARGE}
            \begin{equation*}
                \underbrace{p \Rightarrow q}_{p \textit{implica} q}
            \end{equation*}
        \end{LARGE}
        Se $p$ \textit{implica} $q$ e $q$ \textit{implica} $p$ si dicono \textit{equivalenti}\newline
        \begin{LARGE}
            \begin{equation*}
                p\ \iff\ q
            \end{equation*}
        \end{LARGE}
        \subsection{Quantificatori}
            \begin{itemize}
                \item $\forall$ - per ogni
                \item $\exists$ - esiste
                \item $\exists!$ - esiste ed è unico
                \item $\nexists$ - non esiste
            \end{itemize}
        \subsection{Definizioni, teoremi ed enunciati}
        \paragraph{Definizione:\\ }
            Descrizione univoca di un oggetto.
        \paragraph{Teorema:\\ }
            Affermazione che coinvolge oggetti già definiti
        \paragraph{Enunciato:\\ }
            Un affermazione  da dimostrare composta da un'\textit{ipotesi} e da una \textit{tesi}.
        \hfill \break
        \paragraph{Dimostrazione:\\ }
            Una dimostrazione è l'insieme dei passaggi logici e di calcolo che verificano un enunciato.
        