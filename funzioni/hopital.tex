\subsection{Teorema de L' Hopital}
Sia I un intervallo o un intervallo forato di $\mathbb{R}$, sia  \(c \in [inf_I, sup_I], f, g : I \rightarrow \mathbb{R}\), derivabili in \(I \backslash \{c\} \)
Supponiamo g e g' siano diversi da 0 in \(I \backslash \{c\} \). \\ 
Se 
\begin{equation*}
    \lim_{x \rightarrow c} f(x)= \lim_{x \rightarrow c} g(x)= 0
\end{equation*}
oppure 
\begin{equation*}
    \lim_{x \rightarrow c} f(x) =  \pm \infty,  \lim_{x \rightarrow c} g(x)=  \pm \infty
\end{equation*}
e se 
\begin{equation*}
    \exists \lim_{x \rightarrow c}  \displaystyle\frac{f(x)}{g(x)} = l
\end{equation*}
Allora 
\begin{equation*}
    \lim_{x \rightarrow c}  \displaystyle\frac{f'(x)}{g'(x)} = l
\end{equation*}