\section{Funzioni}
\subsection{Definizione}
Siano $A$ e $B$ due insiemi, $A,B \neq \varnothing$, chiamiamo \textbf{funzione} $f$ da $A$ a $B$
una legge che ad ogni elemento di $A$ associa uno ed un solo elemento di $B$
\[
    \forall x \in A, \exists! y \in B\ |\ y = f(x)    
\]
\break \hfill
\begin{Large}
\[
    f: A \rightarrow B
\]
\end{Large}
\begin{center}
    \hspace{0.75cm}$\Downarrow$ \hspace{0.75cm} $\Downarrow$\\
    \begin{small}
    \qquad \quad \quad \{Dominio\} \{Codominio\}
    \end{small}
\end{center}
\subsubsection*{Esempio}
\begin{Large}
\[
    f: \mathbb{R} \rightarrow \mathbb{R}    
\]
\[
    f(x) = 2x    
\]
\end{Large}
oppure
\begin{Large}
\[
    f: \mathbb{R} \rightarrow \mathbb{R}
\]
\[
    x \mapsto 2x
\]
\end{Large}

\subsection{Grafico}
Il grafico di una funzione è un sottoinsime del \textbf{prodotto cartesiano} tra $A$ e $B$.
\begin{Large}
\[
    f:A \rightarrow B
\]
\[
    Graf_{f}= \{(x,y) \in A \times B\ |\ y=f(x)\}
\]
\end{Large}
\subsection{Iniettività e Suriettività}
\subsubsection*{Definizione}
Diciamo che $f:A\rightarrow B$ e \textbf{suriettiva} se l'immagine (sottoinsime del codominio) coincide
con il codominio

