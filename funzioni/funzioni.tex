\section{Funzioni}
\subsection{Definizione}
Siano $A$ e $B$ due insiemi, $A,B \neq \varnothing$, chiamiamo \textbf{funzione} $f$ da $A$ a $B$
una legge che ad ogni elemento di $A$ associa uno ed un solo elemento di $B$
\begin{Large}
\[
    \forall x \in A, \exists! y \in B\ |\ y = f(x)    
\]
\vspace{-1em}
\[
    f: A \text{\quad} \rightarrow \text{\quad} B
\]
\end{Large}
\vspace{-6.5ex}
\begin{center}
    \hspace{2em}$\Downarrow$ \hspace{5em} $\Downarrow$\\
    \begin{small}
    \qquad \{Dominio\} \quad \{Codominio\}
    \end{small}
\end{center}
\subsubsection*{Esempio}
\begin{Large}
\[
    f: \mathbb{R} \rightarrow \mathbb{R}    
\]
\[
    f(x) = 2x    
\]
\end{Large}
oppure
\begin{Large}
\[
    f: \mathbb{R} \rightarrow \mathbb{R}
\]
\[
    x \mapsto 2x
\]
\end{Large}

\subsection{Grafico}
Il grafico di una funzione è un sottoinsime del \textbf{prodotto cartesiano} tra $A$ e $B$.
\begin{Large}
\[
    f:A \rightarrow B
\]
\[
    Graf_{f}= \{(x,y) \in A \times B\ |\ y=f(x)\}
\]
\end{Large}
\subsection{Iniettività e Suriettività}
\subsubsection*{Definizione}
Diciamo che $f:A\rightarrow B$ è \textbf{suriettiva} se l'immagine (sottoinsime del codominio) coincide
con il codominio
\begin{large}
\[
    \forall\ y \in B, \exists\ x \in A\ |\ y = f(x)    
\]
\[
    f:A \rightarrow B
\]
\end{large}
\vspace{-7ex}
\begin{center}
\begin{small}
    \hspace{0.5cm}$su$
\end{small}
\end{center}
\subsubsection*{Definizione}
Diciamo che $f:A\rightarrow B$ è \textbf{iniettiva} se:
\begin{large}
\[
    \forall\ y \in\ f(A), \exists! x\ \in\ A\ |\ y = f(x)    
\]
\[
    f:A \rightarrow B    
\]
\end{large}
\vspace{-7ex}
\begin{center}
\begin{small}
    \hspace{0.5cm}$|\text{\hspace{-0.1cm}}-\text{\hspace{-0.1cm}}|$
\end{small}
\end{center}
\subsubsection*{Definizione}
Diciamo che $f:A\rightarrow B$ è \textbf{biunivoca} (o biiettiva o invertibile) se è sia suriettiva che iniettiva.

\subsubsection*{Esempio}
\[f: \mathbb{R} \rightarrow \mathbb{R}\]
\[f(x) = x^{2}\]

$f$ non è suriettiva perchè non assume valori negativi, quindi non copre tutto il codominio $\mathbb{R}$.
$f$ non è iniettiva perchè per ogni $x > 0$ esiste più di una $y$

\subsubsection*{Definizione}
Sia $f: A\rightarrow B$ biunivoca, chiamiamo funzione \textbf{inversa} di $f$ (indicandola con $f^{-1}$ la funzione:
\begin{Large}
\[
    f:B \rightarrow A)
\]
\[
    \forall\ y\ \in\ B,\ f^{-1}(y) = y = f(x)    
\]
\end{Large}
\subsection{Composizione di funzioni}
\subsubsection*{Definizione}
Siano $X,Y,Z,W$ insiemi $\neq \oslash$,
\[f:X \rightarrow Y, g:Z\rightarrow W\ |\ f(X) \subseteq Z\]
chiamiamo \textbf{funzione composta} $g \circ f$ la funzione:
\[
    g \circ f: X \rightarrow W, (g \circ f)(x) = g(f(x)),\ \forall\ x \in X 
\]