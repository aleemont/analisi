\section{Funzioni}
\subsection{Definizione}
Siano $A$ e $B$ due insiemi, $A,B \neq \varnothing$, chiamiamo \textbf{funzione} $f$ da $A$ a $B$
una legge che ad ogni elemento di $A$ associa uno ed un solo elemento di $B$
\begin{Large}
\[
    \forall x \in A, \exists! y \in B\ |\ y = f(x)    
\]
\vspace{-1em}
\[
    f: A \text{\quad} \rightarrow \text{\quad} B
\]
\end{Large}
\vspace{-6.5ex}
\begin{center}
    \hspace{2em}$\Downarrow$ \hspace{5em} $\Downarrow$\\
    \begin{small}
    \qquad \{Dominio\} \quad \{Codominio\}
    \end{small}
\end{center}
\subsubsection*{Esempio}
\begin{Large}
\[
    f: \mathbb{R} \rightarrow \mathbb{R}    
\]
\[
    f(x) = 2x    
\]
\end{Large}
oppure
\begin{Large}
\[
    f: \mathbb{R} \rightarrow \mathbb{R}
\]
\[
    x \mapsto 2x
\]
\end{Large}

\subsection{Grafico}
Il grafico di una funzione è un sottoinsime del \textbf{prodotto cartesiano} tra $A$ e $B$.
\begin{Large}
\[
    f:A \rightarrow B
\]
\[
    Graf_{f}= \{(x,y) \in A \times B\ |\ y=f(x)\}
\]
\end{Large}
\subsection{Iniettività e Suriettività}
\subsubsection*{Definizione}
Diciamo che $f:A\rightarrow B$ è \textbf{suriettiva} se l'immagine (sottoinsime del codominio) coincide
con il codominio
\begin{large}
\[
    \forall\ y \in B, \exists\ x \in A\ |\ y = f(x)    
\]
\[
    f:A \rightarrow B
\]
\end{large}
\vspace{-7ex}
\begin{center}
\begin{small}
    \hspace{0.5cm}$su$
\end{small}
\end{center}
\subsubsection*{Definizione}
Diciamo che $f:A\rightarrow B$ è \textbf{iniettiva} se:
\begin{large}
\[
    \forall\ y \in\ f(A), \exists! x\ \in\ A\ |\ y = f(x)    
\]
\[
    f:A \rightarrow B    
\]
\end{large}
\vspace{-7ex}
\begin{center}
\begin{small}
    \hspace{0.5cm}$|\text{\hspace{-0.1cm}}-\text{\hspace{-0.1cm}}|$
\end{small}
\end{center}
\subsubsection*{Definizione}
Diciamo che $f:A\rightarrow B$ è \textbf{biunivoca} (o biiettiva o invertibile) se è sia suriettiva che iniettiva.

\subsubsection*{Esempio}
\[f: \mathbb{R} \rightarrow \mathbb{R}\]
\[f(x) = x^{2}\]

$f$ non è suriettiva perchè non assume valori negativi, quindi non copre tutto il codominio $\mathbb{R}$.
$f$ non è iniettiva perchè per ogni $x > 0$ esiste più di una $y$

\subsubsection*{Definizione}
Sia $f: A\rightarrow B$ biunivoca, chiamiamo funzione \textbf{inversa} di $f$ (indicandola con $f^{-1}$ la funzione:
\begin{Large}
\[
    f:B \rightarrow A)
\]
\[
    \forall\ y\ \in\ B,\ f^{-1}(y) = y = f(x)    
\]
\end{Large}
\subsection{Composizione di funzioni}
\subsubsection*{Definizione}
    Siano $X,Y,Z,W$ insiemi $\neq \oslash$,
    \[f:X \rightarrow Y, g:Z\rightarrow W\ |\ f(X) \subseteq Z\]
    chiamiamo \textbf{funzione composta} $g \circ f$ la funzione:
    \begin{Large}
    \[
        g \circ f: X \rightarrow W, (g \circ f)(x) = g(f(x)),\ \forall\ x \in X 
    \]
    \end{Large}
\subsubsection*{Esempio}
\begin{Large}
    \[
        f : \mathbb{R} \rightarrow \mathbb{R}, f(x) = 2x + 1
    \]
    \[
        g : \mathbb{R} \rightarrow \mathbb{R}, g(y) = y^{2}
    \]
    \[
        (g \circ f)(x) = g(f(x)) = g(2x + 1) = (2x + 1)^{2} = 4x^{2} + 4x +1    
    \]
    \[
        (f \circ g)(y) = f(g(y)) = f(y^{2}) = 2y^{2} + 1
    \]
    \[
        g \circ f \neq f \circ g    
    \]
\end{Large}
\subsection{Funzione identità}
\begin{Large}
\[
    f \circ f^{-1} = f^{-1} \circ f = Id    
\]
\[
    Id(x) = x    
\]
\end{Large}
\subsubsection*{Teorema}
Siano $f: X \rightarrow Y, g: Z \rightarrow W$ invertibili, $Y = Z \Rightarrow g \circ f$ è invertibile 
e $(g \circ f)^{-1} = f^{-1} \circ g^{-1}$.

\subsection{Funzioni di una variabile reale}
Sia $c \subseteq \mathbb{R}, f:A \rightarrow \mathbb{R}$
\subsubsection*{Esempi}
\begin{itemize}
\item $f(x) = 2x + 1$
\item $f(x) = |x|$
\item $f(x) = sgn(x)$
\item $f(x) = [x]$
\end{itemize}
\subsubsection*{Definizione}
Sia $f:A\rightarrow \mathbb{R}, A \subseteq \mathbb{R}$, diciamo che $f$ è limitata se lo è $f(A)$.
\begin{Large}
\[
    sup\ f = sup\ f(A)    
\]
\end{Large}
\vspace{-7.5ex}
\begin{center}
\begin{small}
    \hspace{0.7cm}A
\end{small}
\end{center}

\subsubsection*{Definizione}
$f:A \rightarrow \mathbb{R}, A \subseteq \mathbb{R}$
\begin{itemize}
    \item $f$ ha un massimo se lo ha $f(A)$
    \item $f$ ha un minimo se lo ha $f(A)$
\end{itemize}
\begin{Large}
    \[
        \exists\ \bar{x} \in A\ |\ f(x) \leq f(\bar{x}) \forall\ x \in A
    \]

\vspace{-3ex}
\begin{center}
    \hspace{2cm} $\Downarrow$
\end{center}
\vspace{-4ex}
\begin{center}
    \hspace{2cm} $max\ f$
\end{center}
\end{Large}
\subsubsection*{Esempio}
\[
    f:(0,1]\Rightarrow\mathbb{R}    
\]
\[
    f(x) = 2x    
\]
\[
    f((0,1]) = (0,2]    
\]

$(0,2]$ è limitato $\Rightarrow f$ è limitata
\[
    inf\ f = 0  \text{\hspace{3ex} sup\ f = 2}
\]
\vspace{-4.5ex}
\begin{small}
\[(0,1] \text{\hspace{9ex}} (0,1]\]
\end{small}
\[
    max\ f = 2  \text{\hspace{3ex}} \nexists min\ f = 2
\]
\vspace{-4ex}
\begin{small}
\[(0,1] \text{\hspace{9ex}} (0,1]\]
\end{small}

\subsection{Funzioni crescenti e decrescenti}
\subsubsection*{Definizione}
$f:A\rightarrow \mathbb{R}, A \subseteq \mathbb{R}$ \\
Diciamo che $f$ è \textbf{crescente} ($f \nearrow$) se:
\begin{Large}
    \[
        \forall x_{1},x_{2}, x_{1}\leq x_{2} \Rightarrow f(x_{1}\leq f(x_{2}))
    \]
\end{Large}
\subsubsection*{Definizione}
$f:A \rightarrow \mathbb{R}, A \subseteq \mathbb{R}$ \\
Diciamo che $f$ è \textbf{decrescente} ($f \searrow$) se:
\begin{Large}
    \[
        \forall x_{1},x_{2}, x_{1}\leq x_{2} \Rightarrow f(x_{1}\geq f(x_{2}))    
    \]
\end{Large}
\subsubsection*{Definizione}
$f$ è strettamente crescente o decrescente se valgono le disuguaglianze strette.