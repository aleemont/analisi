\subsection{Radice n-esima}
    Sia \begin{Large}$a \in \mathbb{R}_{+}, n \in \mathbb{N}^{\star} \setminus \{1\}$ \end{Large}, diciamo che \begin{Large}$x \in \mathbb{R}_{+}$\end{Large} è \textbf{radice n-esima} di $a$
    se \begin{Large}$x^{n} = a$\end{Large} e lo indichiamo con \begin{Large} $\sqrt[{n}]{a}$ \end{Large}
    \subsubsection*{Proprietà}
        \begin{Large}
            \begin{enumerate}
                \item $\sqrt[{n}]{ab} = \sqrt[{n}]{a} \cdot \sqrt[{n}]{b}$ 
                \item $\sqrt[{n}]{a+b} \leq \sqrt[{n}]{a} + \sqrt[{n}]{b}$
                \item Se $n$ è pari: $\sqrt[{n}]{x^{m}} = |x|$
            \end{enumerate}
        \end{Large}